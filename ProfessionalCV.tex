% LaTeX resume using res.cls
\documentclass[margin]{res}
%\usepackage{helvetica} % uses helvetica postscript font (download helvetica.sty)
%\usepackage{newcent}   % uses new century schoolbook postscript font 
\setlength{\textwidth}{5.4 in} % set width of text portion
\usepackage[utf8]{inputenc} % this package allows dashes to show up
\linespread{1.25}
\usepackage{amsmath}
\usepackage{amssymb}
\usepackage{fontawesome}
\usepackage{hyperref}
%\usepackage{geometry}
\addtolength{\textheight}{1.75in}


\begin{document}

% Center the name over the entire width of resume:
 \moveleft.5\hoffset\centerline{\huge\bf Benjamin Fattori}
% Draw a horizontal line the whole width of resume:
 \moveleft\hoffset\vbox{\hrule width\resumewidth height 1pt}\smallskip
% address begins here
% Again, the address lines must be centered over entire width of resume:
 \moveleft.5\hoffset\centerline{Toronto, ON } % address
 \moveleft.5\hoffset\centerline{Canada} %
\moveleft.5\hoffset\centerline{ben.fattori@mail.utoronto.ca}
\moveleft.5\hoffset\centerline{\href{https://github.com/fattorib}{\Large\faGithubSquare} \href{https://www.linkedin.com/in/benjamin-fattori-b7b6b11a9/}{\Large\faLinkedinSquare}}
\begin{resume}
 
%\section{OBJECTIVE}  To understand what I am, where I come from, and where I am going. 
\section{EDUCATION} 
{\sl \textbf{Honours Bachelor of Science}} \hfill 2015 - 2020 \\
University of Toronto  \\
Mathematics Specialist, Physics Major \\
Distinction 
                      % \sl will be bold italic in New Century Schoolbook (or
	              % any postscript font) and just slanted in
		      %	Computer Modern (default) font
		      
\section{FURTHER EDUCATION} 
{\sl \textbf{Applied Machine Learning in Python}} \hfill July 2020 \\
University of Michigan, Coursera Course\\
Worked with many popular machine learning models using various Python libraries (e.g. Scikit-learn, Numpy, Pandas, and Matplotlib)

\section{EMPLOYMENT}
{\sl \textbf{Undergraduate Research Assistant}} \hfill February, 2019 - June, 2019 \\
Department of Mathematics, University of Toronto  \\ 
$\cdot$ Helped design a novel model of oscillatory behaviour in the    reward pathway of the mammalian brain\\ 
$\cdot$ Wrote extensive code in MATLAB to implement, analyze and make predictions about the model\\
$\cdot$ Applied techniques learned in differential equations and math modelling courses \\
$\cdot$ Research is ongoing with goal of publishing results in 2020\\
\\
{\sl \textbf{Art of Problem Solving - Grader}} \hfill April, 2019 - Present \\
Toronto, ON \\
$\cdot$ Provide clear and detailed feedback to student submissions\\
$\cdot$ Courses graded include number theory, combinatorics, algebra, and calculus 

\section{SKILLS}
    $\cdot$ \textbf{Python}: Experienced; Used alongside Scikit-learn and NumPy, in computational physics courses, mathematics courses and pure CS courses\\
    $\cdot$ \textbf{MATLAB}: Experienced; Used in applied math research for simulations, implementing mathematical models and solving differential equations\\
    $\cdot$ \textbf{PyTorch}: Experienced\\
    $\cdot$ \textbf{SQL Server}: Junior Developer\\
    $\cdot$ \textbf{C\#}: Junior Developer; Some experience with Microsoft Azure
\section{OTHER SKILLS}
    $\cdot$ \textbf{Communication}: Excellent written and verbal skills\\
    $\cdot$ \textbf{Strong analytical skills}: Strong quantitative and analytical reasoning skills\\
    $\cdot$ \textbf{Teamwork}: Used to working in groups and collaborating with others on research projects and coursework \\
    $\cdot$ \textbf{Quick Learner}: Ability to pick up difficult concepts and apply them to situations quickly\\
\end{resume}
\end{document}




