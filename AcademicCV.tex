% LaTeX resume using res.cls
\documentclass[margin]{res}
%\usepackage{helvetica} % uses helvetica postscript font (download helvetica.sty)
%\usepackage{newcent}   % uses new century schoolbook postscript font 
\setlength{\textwidth}{5.4 in} % set width of text portion
\usepackage[utf8]{inputenc} % this package allows dashes to show up
\linespread{1.25}
\usepackage{amsmath}
\usepackage{amssymb}
\usepackage{fontawesome}
\usepackage{hyperref}
%\usepackage{geometry}
\addtolength{\textheight}{1.75in}


\begin{document}

% Center the name over the entire width of resume:
 \moveleft.5\hoffset\centerline{\huge\bf Benjamin Fattori}
% Draw a horizontal line the whole width of resume:
 \moveleft\hoffset\vbox{\hrule width\resumewidth height 1pt}\smallskip
% address begins here
% Again, the address lines must be centered over entire width of resume:
 \moveleft.5\hoffset\centerline{Toronto, ON } % address
 \moveleft.5\hoffset\centerline{Canada} %
\moveleft.5\hoffset\centerline{ben.fattori@mail.utoronto.ca}
\moveleft.5\hoffset\centerline{\href{https://github.com/fattorib}{\Large\faGithubSquare} \href{https://www.linkedin.com/in/benjamin-fattori-b7b6b11a9/}{\Large\faLinkedinSquare}}
\begin{resume}
 
%\section{OBJECTIVE}  To understand what I am, where I come from, and where I am going. 
\section{EDUCATION} 
{\sl Honours Bachelor of Science} \hfill 2015 - 2020 \\
University of Toronto  \\
Mathematics Specialist, Physics Major \\
CGPA: 3.30/4.00 - Distinction (3.63/4.00 past two years)
                      % \sl will be bold italic in New Century Schoolbook (or
	              % any postscript font) and just slanted in
		      %	Computer Modern (default) font
\section{RESEARCH EXPERIENCE}
{\sl Undergraduate Research Assistant} \hfill March, 2019 - June, 2019 \\
Department of Mathematics, University of Toronto  \\ 
Supervisor: Professor Adam Stinchcombe  \\ 
\\
$\cdot$ Helped design a novel model of oscillatory behaviour in the reward pathway of the mammalian brain\\ 
$\cdot$ Applied techniques learned in differential equations and math modelling courses \\
$\cdot$ Used MATLAB and XPPAUT to analyze the behaviour of the system
\section{IN PREPARATION}
{\sl A Model of the Dopamine Regulated Circadian Oscillator}\\
Adam. R. Stinchcombe, Martin Ralph, Cameron Martin, Benjamin Fattori

\section{TALKS GIVEN}    
        $\cdot$ {\sl The Density of Discriminants of Quartic Rings and Fields} \hfill November 2019 \\
	    MAT477: Introduction to Arithmetic Invariant Theory, University of Toronto\\
	    $\cdot$ {\sl Rings and Ideal Parametrized by Binary $n$-ic forms} \hfill October 2019 \\
	    MAT477: Introduction to Arithmetic Invariant Theory, University of Toronto \\
	    $\cdot$ {\sl Computing the K-Theory of $C(\mathbb{R}\text{P}^2)$} \hfill February 2019 \\
	    George Elliott’s K-theory for $C^\star$-algebras course University of Toronto 
\section{FURTHER EDUCATION} 
{\sl \textbf{Applied Machine Learning in Python}} \hfill July 2020 \\
University of Michigan, Coursera Course\\
Worked with many popular machine learning models using various Python libraries (e.g. Scikit-learn, Numpy, Pandas, and Matplotlib)     
\section{COMPUTER SKILLS}
    $\cdot$ \textbf{Python}: Experienced; Used alongside Scikit-learn and NumPy, in computational physics courses, mathematics courses and pure CS courses\\
    $\cdot$ \textbf{MATLAB}: Experienced; Used in applied math research for simulations, implementing mathematical models and solving differential equations\\
    $\cdot$ \textbf{PyTorch}: Experienced\\
    $\cdot$ \textbf{XPPAUT}: Comfortable; Used in math research for producing bifurcation plots and further examining the behaviour of dynamical systems\\
    $\cdot$ \textbf{$\LaTeX$}: Experienced; Used for typesetting course notes and problem sets since third year\\
\end{resume}
\end{document}




